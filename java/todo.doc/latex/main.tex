%% -*- eval: (flyspell-mode 1); eval: (ispell-change-dictionary "english"); -*-

\documentclass[10pt,onecolumn]{report}
\usepackage[utf8]{inputenc}
\usepackage[T1]{fontenc}

\usepackage{xspace}
\usepackage{fancyvrb}

\usepackage{graphicx}

\usepackage{geometry}
\geometry{hmargin=2cm,vmargin=2cm}

\usepackage{framed}

\usepackage{enumitem}
\setlistdepth{7}

\usepackage{hyperref}
\hypersetup{colorlinks=true}


\begin{document}


\begin{titlepage}

  \vspace{4cm}

  \hrule
  \begin{center}
    \Huge{\textit{Export java todos' content in \LaTeX}}\\
    \vspace{0.5cm}
    \huge\textbf{Technical guide \& Example}\\
  \end{center}
  \hrule

  \vspace{6cm}
  \large{\emph{Authors:} Alexandre Honorat}
  \vspace{1cm}
  \center{\today}

\end{titlepage}


\tableofcontents

\chapter{Usage}

\section{Simplified instructions}

Start by copying this module into your current \textsf{Maven} project, and reference it
from your parent \verb!pom!. Then in the module from where you want to export \verb!todo! 
tags, copy the following block into the \verb!pom!. Don't forget to change the paths.

\begin{framed}
\begin{verbatim}
<reporting>
<plugins>
<plugin>
<groupId>org.apache.maven.plugins</groupId>
<artifactId>maven-javadoc-plugin</artifactId>
<version>2.10.3</version>
<reportSets>
<reportSet>
<id>tex.todo.doc</id>
<reports>
<report>javadoc</report>
</reports>
<configuration>
<show>private</show>

<!-- Relative to ./target -->
<destDir>/path/to/todo.doc/latex/appendix</destDir>

<tagletpath>${doc.jar.location}</tagletpath>
<taglet>todo.doctaglet.latexexporter.ToDoTaglet</taglet>
<useStandardDocletOptions>false</useStandardDocletOptions>

<docletPath>/path/to/latexexporter-0.0.1-SNAPSHOT.jar</docletPath>

<doclet>todo.doctaglet.latexexporter.TodoLatexDoclet</doclet>
</configuration>
</reportSet>
</reportSets>
</plugin>
</plugins>
</reporting>
\end{verbatim}
\end{framed}

Then just put some \verb!todo! tags in your documented module, and call
\verb!mvn site! in it ! And recompile this file, of course.

\section{More complicated stuff}

\paragraph{}
First thing: \verb!todo! tags are collected for classes, methods and 
constructors only.

\paragraph{}
Second thing: the generator transform a little bit the text present in the 
\verb!todo! comment. It will escape underscores and it will transform
\verb!{@code ...}! parts into \verb!\texttt{...}! formatted text. Also some
others replacements are made:
\begin{itemize}
\item \verb!&amp;! by \&;
\item \verb!&lt! by $<$;
\item \verb!&gt! by $>$.
\end{itemize}

\paragraph{}
Third thing, the full prefixed name of a method can be quite annoying so the
prefix part will be removed accordingly to the variable: \verb!QUALIFIER_PREFIX!
("todo.doctaglet." by default) in the file \verb!TodoLatexDoclet.java!, you have
to change it to escape your own package prefixes.

\paragraph{}
A classic documentation will also be generated when running the \verb!mvn site!
command. If, when writing a \verb!todo! tag, you also put just after it the line :
\verb!<span class="todo">--&gt;&gt;</span>! and you copy the file 
\verb!javadoc/stylesheet.css! at the same location in your documented module to
export, you will have a nice red arrow to easily identify such tags in your
documentation. 

\paragraph{}
Finally, a little example:
\begin{framed}
\begin{verbatim}
/**
 * @todo <span class="todo">--&gt;&gt;</span>
 * This is my todo. {@code Such wow.}
 *
 */
public void doNothing(int param) {}
\end{verbatim}
\end{framed}
The \LaTeX generated code will be:
\begin{framed}
\begin{verbatim}
\subsection*{my.package.doNothing(int)}

 This is my todo. \texttt{Such wow.}
\end{verbatim}
\end{framed}

\subsection*{my.package.doNothing(int)}

 This is my todo. \texttt{Such wow.}




%%

\appendix
\chapter{Generated todos}

\IfFileExists{./appendix/todo.tex}
{
  \input{appendix/todo.tex}
}{
  \section{To do:}
  This section will be filled once you'll run the command \texttt{mvn site} in
  your project main folder. It will be generated directly from the commented
  sources containing a @todo tag.
}

%%


\end{document}
